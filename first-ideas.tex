\documentclass{article}

\newcommand{\hide}[1]{}
\usepackage{amsfonts}
\usepackage{amsmath}
\usepackage{amssymb}
\usepackage{amsthm}
\usepackage{datetime}
\usepackage{enumerate}
\usepackage{graphicx}
\usepackage{hyperref}
\usepackage{mathpartir}
\usepackage{quiver}
\usepackage{stmaryrd}
\usepackage{tikz-cd}
\usepackage{tikz}

\newtheorem{thm}{Theorem}
\newtheorem{prop}[thm]{Proposition}
\newtheorem{cor}[thm]{Corollary}
\newtheorem{lem}[thm]{Lemma}
\newtheorem{conj}[thm]{Conjecture}
\newtheorem{claim}[thm]{Claim}

\theoremstyle{definition}
\newtheorem{defn}[thm]{Definition}
\newtheorem{example}[thm]{Example}
\newtheorem{question}[thm]{Question}

\theoremstyle{remark}
\newtheorem{remark}[thm]{Remark}

\newcommand{\MCA}{\ensuremath{\mathcal A}}
\newcommand{\MCB}{\ensuremath{\mathcal B}}
\newcommand{\MCC}{\ensuremath{\mathcal C}}
\newcommand{\MCD}{\ensuremath{\mathcal D}}
\newcommand{\MCE}{\ensuremath{\mathcal E}}
\newcommand{\MCF}{\ensuremath{\mathcal F}}
\newcommand{\MCG}{\ensuremath{\mathcal G}}
\newcommand{\MCH}{\ensuremath{\mathcal H}}
\newcommand{\MCI}{\ensuremath{\mathcal I}}
\newcommand{\MCJ}{\ensuremath{\mathcal J}}
\newcommand{\MCK}{\ensuremath{\mathcal K}}
\newcommand{\MCL}{\ensuremath{\mathcal L}}
\newcommand{\MCM}{\ensuremath{\mathcal M}}
\newcommand{\MCN}{\ensuremath{\mathcal N}}
\newcommand{\MCO}{\ensuremath{\mathcal O}}
\newcommand{\MCP}{\ensuremath{\mathcal P}}
\newcommand{\MCQ}{\ensuremath{\mathcal Q}}
\newcommand{\MCR}{\ensuremath{\mathcal R}}
\newcommand{\MCS}{\ensuremath{\mathcal S}}
\newcommand{\MCT}{\ensuremath{\mathcal T}}
\newcommand{\MCU}{\ensuremath{\mathcal U}}
\newcommand{\MCV}{\ensuremath{\mathcal V}}
\newcommand{\MCW}{\ensuremath{\mathcal W}}
\newcommand{\MCX}{\ensuremath{\mathcal X}}
\newcommand{\MCY}{\ensuremath{\mathcal Y}}
\newcommand{\MCZ}{\ensuremath{\mathcal Z}}

\newcommand{\BA}{\ensuremath{\mathbb A}}
\newcommand{\BB}{\ensuremath{\mathbb B}}
\newcommand{\BC}{\ensuremath{\mathbb C}}
\newcommand{\BD}{\ensuremath{\mathbb D}}
\newcommand{\BE}{\ensuremath{\mathbb E}}
\newcommand{\BF}{\ensuremath{\mathbb F}}
\newcommand{\BG}{\ensuremath{\mathbb G}}
\newcommand{\BH}{\ensuremath{\mathbb H}}
\newcommand{\BI}{\ensuremath{\mathbb I}}
\newcommand{\BJ}{\ensuremath{\mathbb J}}
\newcommand{\BK}{\ensuremath{\mathbb K}}
\newcommand{\BL}{\ensuremath{\mathbb L}}
\newcommand{\BM}{\ensuremath{\mathbb M}}
\newcommand{\BN}{\ensuremath{\mathbb N}}
\newcommand{\BO}{\ensuremath{\mathbb O}}
\newcommand{\BP}{\ensuremath{\mathbb P}}
\newcommand{\BQ}{\ensuremath{\mathbb Q}}
\newcommand{\BR}{\ensuremath{\mathbb R}}
\newcommand{\BS}{\ensuremath{\mathbb S}}
\newcommand{\BT}{\ensuremath{\mathbb T}}
\newcommand{\BU}{\ensuremath{\mathbb U}}
\newcommand{\BV}{\ensuremath{\mathbb V}}
\newcommand{\BW}{\ensuremath{\mathbb W}}
\newcommand{\BX}{\ensuremath{\mathbb X}}
\newcommand{\BY}{\ensuremath{\mathbb Y}}
\newcommand{\BZ}{\ensuremath{\mathbb Z}}

\ddmmyyyydate
\renewcommand{\dateseparator}{-}

\makeatletter
\def\@maketitle{%
  \newpage
  \null
  \vskip 2em%
  \begin{center}%
  \let \footnote \thanks
    {\large \textbf \@title \par}%
    \vskip 1.5em%
    {
      \lineskip .5em%
      \begin{tabular}[t]{c}%
        \@author
      \end{tabular}\par}%
    \vskip 1em%
    {Draft of \@date}%
  \end{center}%
  \par
  \vskip 1.5em}
\makeatother

\newcommand{\todo}[1]{[Todo: #1]}

\DeclareMathOperator*{\colim}{colim}

\newcommand{\op}{^{\mathrm{op}}}
\newcommand{\Kl}{\mathrm{Kl}}
\newcommand{\Surj}{\mathrm{Surj}}
\newcommand{\Set}{\mathbf{Set}}
\newcommand{\FinSet}{\mathbf{FinSet}}
\newcommand{\MainCategory}{\widetilde{\Kl}(T)}
\newcommand{\CCat}{\mathcal{C}}
\newcommand{\DCat}{\mathcal{D}}
\newcommand{\GCat}{\mathcal{G}}
\newcommand{\El}{\mathrm{El}}

\newcommand{\preT}{[\MCT\op, \Set]}
\newcommand{\bangle}[1]{\langle #1 \rangle}
\newcommand\sem[1]{\llbracket #1 \rrbracket}
\newcommand\opsem[1]{\llparenthesis #1 \rrparenthesis}
\newcommand{\bind}{\mathrel{\scalebox{0.5}[1]{$>\!>=$}}}
\newcommand{\CP}{\overline{P}}

\newcommand{\mult}{\phi}
\newcommand{\unit}{\eta}

\newcommand{\Idm}[1]{\mathsf{Idm}(#1)}

\newcommand{\image}{\mathsf{image}}
\newcommand\with{\oplus}

\title{Notes on idioms}
\author{Jack Liell-Cock, Theo Wang}
\date{\today}

\begin{document}

\maketitle

The following can be defined at the generality of monoidal categories. However, this seems like overkill for the applications in mind.
We should at the very most consider symmetric monoidal closed categories but probably easiest to consider Cartesian closed categories.

\begin{defn}
  Let $\CCat$ be a monidal category. An \emph{idiom} is a pointed lax monoidal endofunctor on $\CCat$.
  That is, an idiom $(J, \unit, \varepsilon, \mult)$ consists of:
  \begin{itemize}
    \item An endofunctor $J: \CCat \to \CCat$,
    \item A family of maps $\unit_A: A \to JA$ natural in $A \in \CCat$,
    \item A map $\varepsilon: 1 \to J1$,
    \item A family of maps $\mult_{A,B}: JA \times JB \to J(A\times B)$ natural in $A,B \in \CCat$,
  \end{itemize}
  such that the following diagrams commute.
  \begin{mathpar}
    \begin{tikzcd}[column sep=large]
      {(JA\times JB)\times JC} & {J(A\times B) \times JC} & {J((A\times B)\times C)} \\
      {JA\times (JB\times JC)} & {JA\times J(B\times C)} & {J(A\times (B\times C))}
      \arrow["{\phi_{A,B}\times JC}", from=1-1, to=1-2]
      \arrow["{\alpha_{JA,JB,JC}}"', from=1-1, to=2-1]
      \arrow["{\phi_{A\times B, C}}", from=1-2, to=1-3]
      \arrow["{J\alpha_{A,B,C}}", from=1-3, to=2-3]
      \arrow["{JA\times \phi_{B,C}}"', from=2-1, to=2-2]
      \arrow["{\phi_{A,B\times C}}"', from=2-2, to=2-3]
    \end{tikzcd}
    \and
    \begin{tikzcd}[column sep=large]
      {1\times JA} & {J1 \times JA} \\
      JA & {J(1\times A)}
      \arrow["{\varepsilon \times JA}", from=1-1, to=1-2]
      \arrow["{\lambda_{JA}}"', from=1-1, to=2-1]
      \arrow["{\phi_{A,1}}", from=1-2, to=2-2]
      \arrow["{J\lambda_A}", from=2-2, to=2-1]
    \end{tikzcd}
    \and
    \begin{tikzcd}[column sep=large]
      {A\times B} & {JA\times JB} \\
      & {J(A\times B)}
      \arrow["{\eta_A\times \eta_B}", from=1-1, to=1-2]
      \arrow["{\eta_{A\times B}}"', from=1-1, to=2-2]
      \arrow["{\phi_{A,B}}", from=1-2, to=2-2]
    \end{tikzcd}
    \and
    \begin{tikzcd}
      1 \\
      & J1
      \arrow["\varepsilon"', curve={height=12pt}, from=1-1, to=2-2]
      \arrow["{\eta_1}", curve={height=-12pt}, from=1-1, to=2-2]
    \end{tikzcd}
  \end{mathpar}
  The coherences imply that $\varepsilon = \unit_1$, so we often elide it.
\end{defn}

Idioms can be categorically presented as \emph{strong} lax monoidal functors~\cite{mcbride-applicatives}.
The following proposition links this presentation to the above definition. It seems like folklore, but I haven't seen it written down anywhere.
The benefits of the above definition is that there are way fewer coherence diagrams to check.
Also, it aligns with the data of applicatives in Haskell way more closely.

\begin{prop}
  Let $\CCat$ be a monidal category. The following data are equivalent:
  \begin{enumerate}
    \item A lax monoidal endofunctor $(F, \varepsilon, \phi): \CCat \to \CCat$ with coherent pointedness $\eta_A: A \to FA$.
      In particular, the following diagrams commute.
      \begin{mathpar}
        \begin{tikzcd}[column sep=large]
          {A\otimes B} & {JA\otimes JB} \\
          & {J(A\otimes B)}
          \arrow["{\eta_A\otimes \eta_B}", from=1-1, to=1-2]
          \arrow["{\eta_{A\otimes B}}"', from=1-1, to=2-2]
          \arrow["{\phi_{A,B}}", from=1-2, to=2-2]
        \end{tikzcd}
        \and
        \begin{tikzcd}
          I \\
          & JI
          \arrow["\varepsilon"', curve={height=12pt}, from=1-1, to=2-2]
          \arrow["{\eta_I}", curve={height=-12pt}, from=1-1, to=2-2]
        \end{tikzcd}
      \end{mathpar}
    \item A lax monoidal endofunctor $(F, \varepsilon, \phi): \CCat \to \CCat$ with coherent bistrength $\sigma^l_{A,B}, \sigma^r_{A,B}: FA \otimes B \to F(A\otimes B)$.
      In particular, the following diagrams commute.
      \begin{mathpar}
        \begin{tikzcd}[column sep=large]
          {(A\otimes JB)\otimes JC} & {J(A\otimes B)\otimes JC} & {J((A\otimes B)\otimes C)} \\
          {A\otimes (JB\otimes JC)} & {A\otimes J(B\otimes C)} & {J(A\otimes (B\otimes C))}
          \arrow["{\sigma^l_{A,B}\otimes JC}", from=1-1, to=1-2]
          \arrow["{\alpha_{A,JB,JC}}"', from=1-1, to=2-1]
          \arrow["{\phi_{A\otimes B, C}}", from=1-2, to=1-3]
          \arrow["{J\alpha_{A,B,C}}", from=1-3, to=2-3]
          \arrow["{A\otimes \phi_{B,C}}"', from=2-1, to=2-2]
          \arrow["{\sigma^l_{A,B\otimes C}}"', from=2-2, to=2-3]
        \end{tikzcd}
        \and
        \begin{tikzcd}[column sep=large]
          {(JA\otimes JB)\otimes C} & {J(A\otimes B) \otimes C} & {J((A\otimes B)\otimes C)} \\
          {JA\otimes (JB\otimes C)} & {JA\otimes J(B\otimes C)} & {J(A\otimes (B\otimes C))}
          \arrow["{\phi_{A,B}\otimes C}", from=1-1, to=1-2]
          \arrow["{\alpha_{JA,JB,C}}"', from=1-1, to=2-1]
          \arrow["{\sigma^r_{A\otimes B, C}}", from=1-2, to=1-3]
          \arrow["{J\alpha_{A,B,C}}", from=1-3, to=2-3]
          \arrow["{JA\otimes \sigma^r_{B,C}}"', from=2-1, to=2-2]
          \arrow["{\phi_{A,B\otimes C}}"', from=2-2, to=2-3]
        \end{tikzcd}
      \end{mathpar}
  \end{enumerate}
\end{prop}

\begin{proof}
  In the forwards direction, we define
  \begin{align*}
    \sigma^l_{A,B}: &A \otimes FB \xrightarrow{\eta_A \otimes FB} FA \otimes FB \xrightarrow{\phi_{A,B}} F(A\otimes B), \\
    \sigma^r_{A,B}: &FA \otimes B \xrightarrow{FA \otimes \eta_B} FA \otimes FB \xrightarrow{\phi_{A,B}} F(A\otimes B).
  \end{align*}
  In the reverse direction, we define
  \[
    \eta_A: A \xrightarrow{\lambda_A^{-1}} I \otimes A \xrightarrow{\varepsilon \otimes A} FI \otimes A \xrightarrow{\sigma^r_{I,A}} F(I\otimes A) \xrightarrow{F\lambda_A} FA.
  \]
  Note that this is equivalent to defining
  \[
    \eta_A: A \xrightarrow{\rho_A^{-1}} A \otimes I \xrightarrow{A \otimes \varepsilon} A \otimes FI \xrightarrow{\sigma^l_{A,I}} F(A \otimes I) \xrightarrow{F\rho_A} FA.
  \]
  The coherences lift through correspondingly.
\end{proof}

\paragraph{Enrichment for free}

\begin{prop}
Let $\CCat$ be a Cartesian closed category. Given an idiom $(J_0, \unit, \mult)$ on $\CCat$,
there is a $\CCat$-enriched functor $J : \CCat \to \CCat$ that has same mapping on objects
as $J_0$, and its mapping on hom-objects $J_{A,B}: [A,B] \to [JA, JB]$ is given by the transpose of:
\[
  [A,B] \times JA \xrightarrow{\unit_{[A,B]} \times JA} J[A,B] \times JA \xrightarrow{\mult_{[A,B], A}} J([A,B] \times A) \xrightarrow{J\mathsf{ev}_{A,B}} JB
\]
Moreover, $J$ is enriched pointed lax monoidal, and so is an enriched idiom.
\end{prop}

\begin{proof}
  \todo{For Theo to prove.}
\end{proof}

\paragraph{The idiom category}

The goal of this section is to define a Kleisli-like category for idioms.
In the sense that Kleisli categories of monads capture programs with monadic effects, we would like the Kleisli-like category to capture programs with idiomatic effects.

\begin{defn}
  For a Cartesian closed category $\CCat$ considered self-enriched, and an idiom $(J, \unit, \mult)$.
  We define the category $\Idm J$ as the image of the category $\CCat$ under the change of base along $J$.
  That is, the objects of $\Idm J$ are the same as the objects of $\CCat$, and the morphisms are
  \[
    \Idm J [A, B] = J[A,B].
  \]
\end{defn}

\begin{prop}
  The $\CCat$-enriched functor $J: \CCat \to \CCat$ decomposes as
  \[
    J: \CCat \xrightarrow{L} \Idm J \xrightarrow{R} \CCat,
  \]
  where $L$ is defined as
  \begin{mathpar}
    LA = A, \and L_{A,B}(f) = \eta_{[A,B]}(f),
  \end{mathpar}
  and $R$ is defined as
  \begin{mathpar}
    RA = JA, \and R_{A,B}(f) = \lambda a.W(\mathsf{ev})(\phi_{[A,B], A}(f, a)),
  \end{mathpar}
\end{prop}

The category $\Idm J$ seems to be a nice category to do idiomatic computations in, because morphisms are effectful, but cant be sequenced based on their return value.

\subsection{The Idiom Calculus}

\newcommand{\aj}[3]{#1 \vdash_{\mathsf{v}} #2:#3}
\newcommand{\cj}[3]{#1 \vdash_{\mathsf{c}} #2:#3}
\newcommand{\lunit}{{*}}
\newcommand{\app}[2]{{#1 \ #2}}
\newcommand{\lift}[1]{\mathsf{lift}(#1)}

We begin with value judgements (judgements in $\CCat$) which follow the simply-typed lambda calculus:
\begin{mathpar}
  \inferrule{ }{\aj {\Gamma,a:A,\Gamma'} a {A}}
  \and
  \inferrule{ }{\aj {\Gamma} \lunit {1}}
  \and
  \inferrule{\aj \Gamma u {A\times B}}{\aj \Gamma {\pi_1 u} {A}}
  \and
  \inferrule{\aj \Gamma u {A\times B}}{\aj \Gamma {\pi_2 u} {B}}
  \and
  \inferrule{\aj {\Gamma} u {A} \and \aj {\Gamma} v {B}}{\aj {\Gamma} {(u,v)} {A\times B}}
  \and
  \inferrule{\aj {\Gamma,a:A} u B}{\aj \Gamma {\lambda (a:A).u} {A\to B}}
  \and
  \inferrule{\aj \Gamma u {A\to B} \and \aj \Gamma v A}{\aj \Gamma {\app u v} B}
\end{mathpar}
We may \emph{lift} value judgements to idiomatic judgements which corresponds to applying $\unit$ to the hom-object.
\begin{mathpar}
  \inferrule{\aj \Gamma t A}{\cj \Gamma {\lift t} A}
\end{mathpar}
The category $\Idm J$ is symmetric monoidal closed and so we use the linear lambda calculus for computation judgements. We additionally request that
$\mathsf{lift}$ is a strong symmetric monoidal closed functor. For instance,
\begin{mathpar}
  \inferrule{\aj \Gamma u {A\to B} \and \aj \Gamma v A}{\cj \Gamma {{\lift{\app u v}} = \app {\lift u} {\lift v}} B}
\end{mathpar}
We may introduce effects via the following judgement.
\begin{mathpar}
  \inferrule{ }{\cj \bullet {\sigma} A}(\sigma \in JA)
\end{mathpar}

\bibliography{refs}
\bibliographystyle{plainurl}

\end{document}
