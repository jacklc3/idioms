\documentclass{article}

\newcommand{\hide}[1]{}
\usepackage{amsfonts}
\usepackage{amsmath}
\usepackage{amssymb}
\usepackage{amsthm}
\usepackage{datetime}
\usepackage{enumerate}
\usepackage{graphicx}
\usepackage{hyperref}
\usepackage{mathpartir}
\usepackage{quiver}
\usepackage{stmaryrd}
\usepackage{tikz-cd}
\usepackage{tikz}

\newtheorem{thm}{Theorem}
\newtheorem{prop}[thm]{Proposition}
\newtheorem{cor}[thm]{Corollary}
\newtheorem{lem}[thm]{Lemma}
\newtheorem{conj}[thm]{Conjecture}
\newtheorem{claim}[thm]{Claim}

\theoremstyle{definition}
\newtheorem{defn}[thm]{Definition}
\newtheorem{example}[thm]{Example}
\newtheorem{question}[thm]{Question}

\theoremstyle{remark}
\newtheorem{remark}[thm]{Remark}

\newcommand{\MCA}{\ensuremath{\mathcal A}}
\newcommand{\MCB}{\ensuremath{\mathcal B}}
\newcommand{\MCC}{\ensuremath{\mathcal C}}
\newcommand{\MCD}{\ensuremath{\mathcal D}}
\newcommand{\MCE}{\ensuremath{\mathcal E}}
\newcommand{\MCF}{\ensuremath{\mathcal F}}
\newcommand{\MCG}{\ensuremath{\mathcal G}}
\newcommand{\MCH}{\ensuremath{\mathcal H}}
\newcommand{\MCI}{\ensuremath{\mathcal I}}
\newcommand{\MCJ}{\ensuremath{\mathcal J}}
\newcommand{\MCK}{\ensuremath{\mathcal K}}
\newcommand{\MCL}{\ensuremath{\mathcal L}}
\newcommand{\MCM}{\ensuremath{\mathcal M}}
\newcommand{\MCN}{\ensuremath{\mathcal N}}
\newcommand{\MCO}{\ensuremath{\mathcal O}}
\newcommand{\MCP}{\ensuremath{\mathcal P}}
\newcommand{\MCQ}{\ensuremath{\mathcal Q}}
\newcommand{\MCR}{\ensuremath{\mathcal R}}
\newcommand{\MCS}{\ensuremath{\mathcal S}}
\newcommand{\MCT}{\ensuremath{\mathcal T}}
\newcommand{\MCU}{\ensuremath{\mathcal U}}
\newcommand{\MCV}{\ensuremath{\mathcal V}}
\newcommand{\MCW}{\ensuremath{\mathcal W}}
\newcommand{\MCX}{\ensuremath{\mathcal X}}
\newcommand{\MCY}{\ensuremath{\mathcal Y}}
\newcommand{\MCZ}{\ensuremath{\mathcal Z}}

\newcommand{\BA}{\ensuremath{\mathbb A}}
\newcommand{\BB}{\ensuremath{\mathbb B}}
\newcommand{\BC}{\ensuremath{\mathbb C}}
\newcommand{\BD}{\ensuremath{\mathbb D}}
\newcommand{\BE}{\ensuremath{\mathbb E}}
\newcommand{\BF}{\ensuremath{\mathbb F}}
\newcommand{\BG}{\ensuremath{\mathbb G}}
\newcommand{\BH}{\ensuremath{\mathbb H}}
\newcommand{\BI}{\ensuremath{\mathbb I}}
\newcommand{\BJ}{\ensuremath{\mathbb J}}
\newcommand{\BK}{\ensuremath{\mathbb K}}
\newcommand{\BL}{\ensuremath{\mathbb L}}
\newcommand{\BM}{\ensuremath{\mathbb M}}
\newcommand{\BN}{\ensuremath{\mathbb N}}
\newcommand{\BO}{\ensuremath{\mathbb O}}
\newcommand{\BP}{\ensuremath{\mathbb P}}
\newcommand{\BQ}{\ensuremath{\mathbb Q}}
\newcommand{\BR}{\ensuremath{\mathbb R}}
\newcommand{\BS}{\ensuremath{\mathbb S}}
\newcommand{\BT}{\ensuremath{\mathbb T}}
\newcommand{\BU}{\ensuremath{\mathbb U}}
\newcommand{\BV}{\ensuremath{\mathbb V}}
\newcommand{\BW}{\ensuremath{\mathbb W}}
\newcommand{\BX}{\ensuremath{\mathbb X}}
\newcommand{\BY}{\ensuremath{\mathbb Y}}
\newcommand{\BZ}{\ensuremath{\mathbb Z}}

\ddmmyyyydate
\renewcommand{\dateseparator}{-}

\makeatletter
\def\@maketitle{%
  \newpage
  \null
  \vskip 2em%
  \begin{center}%
  \let \footnote \thanks
    {\large \textbf \@title \par}%
    \vskip 1.5em%
    {
      \lineskip .5em%
      \begin{tabular}[t]{c}%
        \@author
      \end{tabular}\par}%
    \vskip 1em%
    {Draft of \@date}%
  \end{center}%
  \par
  \vskip 1.5em}
\makeatother

\newcommand{\todo}[1]{[Todo: #1]}

\DeclareMathOperator*{\colim}{colim}

\newcommand{\op}{^{\mathrm{op}}}
\newcommand{\Kl}{\mathrm{Kl}}
\newcommand{\Surj}{\mathrm{Surj}}
\newcommand{\Set}{\mathbf{Set}}
\newcommand{\FinSet}{\mathbf{FinSet}}
\newcommand{\MainCategory}{\widetilde{\Kl}(T)}
\newcommand{\CCat}{\mathcal{C}}
\newcommand{\DCat}{\mathcal{D}}
\newcommand{\GCat}{\mathcal{G}}
\newcommand{\El}{\mathrm{El}}

\newcommand{\preT}{[\MCT\op, \Set]}
\newcommand{\bangle}[1]{\langle #1 \rangle}
\newcommand\sem[1]{\llbracket #1 \rrbracket}
\newcommand\opsem[1]{\llparenthesis #1 \rrparenthesis}
\newcommand{\bind}{\mathrel{\scalebox{0.5}[1]{$>\!>=$}}}
\newcommand{\CP}{\overline{P}}

\newcommand{\mult}{\phi}
\newcommand{\unit}{\eta}

\newcommand{\Idm}[1]{\mathsf{Idm}(#1)}

\newcommand{\image}{\mathsf{image}}
\newcommand\with{\oplus}

\title{Notes on idioms}
\author{Jack Liell-Cock, Theo Wang}
\date{\today}

\begin{document}

\maketitle

Consider a cartesian closed category $\CCat$.
\begin{defn}
  An \emph{idiom} is a pointed lax monoidal endofunctor on $\CCat$. That is, an idiom $(J, \unit, \varepsilon, \mult)$ consists of:
  \begin{itemize}
    \item An endofunctor $W: \CCat \to \CCat$,
    \item A family of maps $\unit_A: A \to WA$ natural in $A \in \CCat$,
    \item A map $\eta: 1 \to W1$,
    \item A family of maps $\mult_{A,B}: WA \times WB \to J(A\times B)$ natural in $A,B \in \CCat$,
  \end{itemize}
  such that the following diagrams commute.
  \begin{mathpar}
    \begin{tikzcd}[column sep=large]
      {(WA\times WB)\times WC} & {J(A\times B) \times WC} & {J((A\times B)\times C)} \\
      {WA\times (WB\times WC)} & {WA\times J(B\times C)} & {J(A\times (B\times C))}
      \arrow["{\phi_{A,B}\times WC}", from=1-1, to=1-2]
      \arrow["{\alpha_{WA,WB,WC}}"', from=1-1, to=2-1]
      \arrow["{\phi_{A\times B, C}}", from=1-2, to=1-3]
      \arrow["{J\alpha_{A,B,C}}", from=1-3, to=2-3]
      \arrow["{WA\times \phi_{B,C}}"', from=2-1, to=2-2]
      \arrow["{\phi_{A,B\times C}}"', from=2-2, to=2-3]
    \end{tikzcd}
    \and
    \begin{tikzcd}[column sep=large]
      {1\times WA} & {W1 \times WA} \\
      WA & {J(1\times A)}
      \arrow["{\varepsilon \times WA}", from=1-1, to=1-2]
      \arrow["{\lambda_{WA}}"', from=1-1, to=2-1]
      \arrow["{\phi_{A,1}}", from=1-2, to=2-2]
      \arrow["{J\lambda_A}", from=2-2, to=2-1]
    \end{tikzcd}
    \and
    \begin{tikzcd}[column sep=large]
      {A\times B} & {WA\times WB} \\
      & {J(A\times B)}
      \arrow["{\eta_A\times \eta_B}", from=1-1, to=1-2]
      \arrow["{\eta_{A\times B}}"', from=1-1, to=2-2]
      \arrow["{\phi_{A,B}}", from=1-2, to=2-2]
    \end{tikzcd}
    \and
    \begin{tikzcd}
      1 \\
      & W1
      \arrow["\varepsilon"', curve={height=12pt}, from=1-1, to=2-2]
      \arrow["{\eta_1}", curve={height=-12pt}, from=1-1, to=2-2]
    \end{tikzcd}
  \end{mathpar}
  The coherences imply that $\varepsilon = \unit_1$, so we often elide it.
\end{defn}
\begin{prop}
  Let $\CCat$ and $\DCat$ be symmetric monidal categories. The following data are equivalent:
  \begin{enumerate}
    \item A lax monoidal functor $(F, \varepsilon, \phi): \CCat \to \DCat$ with coherent pointedness $\eta_A: A \to FA)$.
      That is, the following diagrams commute.
      \begin{mathpar}
        \begin{tikzcd}[column sep=large]
          {A\times B} & {WA\times WB} \\
          & {J(A\times B)}
          \arrow["{\eta_A\times \eta_B}", from=1-1, to=1-2]
          \arrow["{\eta_{A\times B}}"', from=1-1, to=2-2]
          \arrow["{\phi_{A,B}}", from=1-2, to=2-2]
        \end{tikzcd}
        \and
        \begin{tikzcd}
          1 \\
          & W1
          \arrow["\varepsilon"', curve={height=12pt}, from=1-1, to=2-2]
          \arrow["{\eta_1}", curve={height=-12pt}, from=1-1, to=2-2]
        \end{tikzcd}
      \end{mathpar}
    \item A lax monoidal functor $(F, \varepsilon, \phi): \CCat \to \DCat$ with coherent strength $\sigma_{A,B}: FA \otimes B \to F(A\otimes B)$.
      That is, the following diagrams commute.
      \[
      \begin{tikzcd}[column sep=large]
        {WA\times WB \times C} & {J(A\times B) \times C} \\
        {WA \times J(B\times C)} & {J(A\times B\times C)}
        \arrow["{\phi_{A,B} \times C}", from=1-1, to=1-2]
        \arrow["{WA\times \sigma_{B,C}}"', from=1-1, to=2-1]
        \arrow["{\sigma_{A\times B,C}}", from=1-2, to=2-2]
        \arrow["{\phi_{A,B\times C}}"', from=2-1, to=2-2]
      \end{tikzcd}
      \]
  \end{enumerate}
\end{prop}

\begin{proof}
  In the forwards direction, we define
  \[
    \sigma_{A,B}: FA \otimes B \xrightarrow{FA \otimes \eta_B} FA \otimes FB \xrightarrow{\phi_{A,B}} F(A\otimes B).
  \]
  In the reverse direction, we define
  \[
    \eta_A: A \xrightarrow{\lambda_A^{-1}} I \otimes A \xrightarrow{\varepsilon \otimes A} FI \otimes A \xrightarrow{\sigma_{I,A}} F(I\times A) \xrightarrow{F\lambda_A} FA
  \]
  The coherences lift through correspondingly.
\end{proof}

\section{Enrichment and Decomposition}

Given an idiom $(W_0, \unit, \mult)$, on a Cartesian closed category $\CCat$, there is a $\CCat$-enriched functor $J : \CCat \to \CCat$ that has same mapping on objects
as $W_0$, and its mapping on hom-objects $W_{A,B}: [A,B] \to [WA, WB]$ is given by the transpose of:
\[
  [A,B] \times WA \xrightarrow{\unit_{[A,B]} \times WA} J[A,B] \times WA \xrightarrow{\mult_{[A,B], A}} J([A,B] \times A) \xrightarrow{J\mathsf{ev}_{A,B}} WB
\]

\begin{proof}
  \todo{For Theo}
\end{proof}

\begin{prop}
  The $\CCat$-enriched functor $W: \CCat \to \CCat$ decomposes in the following way:
  \[
    \CCat \xrightarrow{L} \Idm J \xrightarrow{R} \CCat
  \]
  Where $\Idm J$ is the change of base of $\CCat$ along $J$, $L$ is an identity on objects functor that acts on hom-objects via $\unit$, and $R$ sends $A$ to $WA$,
  and $J[A,B]$ to $[WA, WB]$ via the lax monoidal structure of $J$.
  \todo{Make this more explicit.}
\end{prop}

\section{The Idiom Category}


The goal of this section is to define a Kleisli-like category for idioms.
In the sense that Kleisli categories of monads capture programs with monadic effects, we would like the Kleisli-like category to capture programs with idiomatic effects.

\begin{defn}
  For a CCC $\CCat$ considered self-enriched. We define the category $\Idm J$ as the image of the category $\CCat$ under the change of base along $J$. That is,
  the objects of $\Idm J$ are the same as the objects of $\CCat$, and the morphisms are
  \[
    \Idm J [A, B] = J[A,B].
  \]
\end{defn}

Seems to be a nice category to do idiomatic computations in, because morphisms are effectful, but cant be sequenced based on their return value.

\subsection{The Idiom Calculus}

\newcommand{\aj}[3]{#1 \vdash_{\mathsf{v}} #2:#3}
\newcommand{\cj}[3]{#1 \vdash_{\mathsf{c}} #2:#3}
\newcommand{\lunit}{{*}}
\newcommand{\app}[2]{{#1 \ #2}}
\newcommand{\lift}[1]{\mathsf{lift}(#1)}

We begin with value judgements (judgements in $\CCat$) which follow the simply-typed lambda calculus:
\begin{mathpar}
  \inferrule{ }{\aj {\Gamma,a:A,\Gamma'} a {A}}
  \and
  \inferrule{ }{\aj {\Gamma} \lunit {1}}
  \and
  \inferrule{\aj \Gamma u {A\times B}}{\aj \Gamma {\pi_1 u} {A}}
  \and
  \inferrule{\aj \Gamma u {A\times B}}{\aj \Gamma {\pi_2 u} {B}}
  \and
  \inferrule{\aj {\Gamma} u {A} \and \aj {\Gamma} v {B}}{\aj {\Gamma} {(u,v)} {A\times B}}
  \and
  \inferrule{\aj {\Gamma,a:A} u B}{\aj \Gamma {\lambda (a:A).u} {A\to B}}
  \and
  \inferrule{\aj \Gamma u {A\to B} \and \aj \Gamma v A}{\aj \Gamma {\app u v} B}
\end{mathpar}
We may \emph{lift} value judgements to idiomatic judgements which corresponds to applying $J$ to the hom-object.
\begin{mathpar}
  \inferrule{\aj \Gamma t A}{\cj \Gamma {\lift t} A}
\end{mathpar}
The category $\Idm J$ is symmetric monoidal closed and so we use the linear lambda calculus for computation judgements. We additionally request that
$\mathsf{lift}$ is a strong symmetric monoidal closed functor. For instance,
\begin{mathpar}
  \inferrule{\aj \Gamma u {A\to B} \and \aj \Gamma v A}{\cj \Gamma {{\lift{\app u v}} = \app {\lift u} {\lift v}} B}
\end{mathpar}
We may introduce effects via the following judgement.
\begin{mathpar}
  \inferrule{ }{\cj \bullet {\sigma} A}(\sigma \in WA)
\end{mathpar}

\end{document}
